\documentclass{pracamgren}

%\usepackage{color}

\usepackage[T1]{fontenc}
\usepackage{lmodern}

\usepackage[table, svgnames, dvipsnames]{xcolor}
\usepackage{algpseudocode}
\usepackage{algorithm}
\usepackage[framemethod=tikz]{mdframed}
\usepackage{amsmath}
\usepackage{amsfonts}
\usepackage{epsfig}
\usepackage{pstricks}
\usepackage{graphicx}
\usepackage{algorithm}
\usepackage[backend=biber,style=authoryear-comp,sorting=nyt,maxcitenames=2,uniquename=false,uniquelist=false,maxbibnames=5,maxsortnames=1]{biblatex}
\usepackage[font=small,labelfont=bf]{caption}
\usepackage[hidelinks,allcolors=black]{hyperref}
\usepackage{upgreek} 
\usepackage{placeins}
\renewcommand{\arraystretch}{1.1}

\setlength{\parskip}{0pt}

\addbibresource{msc-thesis.bib}

\author{Maciej Manna}

\nralbumu{1065745}

\title{Massively Parallel Pseudo-Spectral DNS and~LES for Particle-Laden Turbulent Flows under Two-Way Momentum Coupling}

\tytul{Masywnie równoległe pseudo-spektralne symulacje DNS i LES przepływów turbulentnych z\,cząstkami przy uwzględnieniu dwustronnego sprzężenia pędu}

\kierunek{computer science}

\opiekun{Dr. hab. Bogdan Rosa \\
Warsaw University of Life Sciences -- SGGW \\
Institute of Information Technology; \\
Institute of Meteorology and Water Management \\ National Research Institute
}

\date{September 2022}

\keywords{obliczeniowa mechanika płynów, przepływ turbulentny, przepływ wielofazowy, krople, mikrofizyka chmur, obliczenia dużej mocy, równania Naviera-Stokesa, homogeniczna izotropowa turbulencja, statystyki zderzeniowe kropel, metoda pseudo-spektralna, bezpośrednia symulacja numeryczna (DNS), metoda dużych wirów (LES), dwustronne sprzeżenie pędu, efekty grawitacyjne}

\keywordsang{computational fluid mechanics, turbulent flow, multiphase flow, droplets, cloud microphysics, high performance computing, Navier-Stokes equations, homogeneous isotropic turbulence, droplet collision statistics, pseudo-spectral method, direct numerical simulation, large-eddy simulation, two-way momentum coupling, gravitational effects}


\begin{document}
\maketitle

\streszczenie{
Przepływy turbulentne z cząstkami fazy rozproszonej zachodzą nieustannie w środowisku naturalnym oraz znajdują liczne zastosowania w procesach przemysłowych.
Ważnym tego przykładem są chmury atmosferyczne, które składają się z ogromnej ilości małych kropel wody i kryształków lodu.
Wyjaśnienie wzajemnego oddziaływania powietrza (płynu) i aerozolu (cząstek) jest istotne w kontekście poznania mechanizmów rządzących powstawaniem i przebiegiem zjawisk atmosferycznych.
Wiedza w tym zakresie jest użyteczna w obszarze meteorologii i służy do opracowywania coraz to bardziej realistycznych modeli powstawania opadu z kropel chmurowych.
To z kolei prowadzi do podniesienia dokładności i sprawdzalności numerycznych prognoz pogody.
Jednym ze sposobów badania przepływów dyspersyjnych są symulacje numeryczne. 
\newline \indent
Powszechnie stosowanym do tego celu narzędziem, zapewniającym wysoką dokładność, są tzw. bezpośrednie symulacje numeryczne (direct numerical simulation, DNS). 
Za pomocą symulacji DNS można pozyskiwać statystyki zderzeniowe cząstek, które to są użyteczne jako parametry w modelach obejmujących większe skale przestrzenne.
Symulacje DNS są~jednak kosztowne obliczeniowo, co ogranicza ich stosowalność do siatek o średnich rozmiarach (${\sim 2048^3}$ węzłów - przy współcześnie dostępnych zasobach obliczeniowych).
DNS nie pozwalają modelować turbulencji charakteryzowanej wyższymi liczbami Reynoldsa ($R_{\lambda} \sim 10^4$), czyli takiej która obserwowana jest w powietrzu atmosferycznym. \newline \indent
Metoda dużych wirów (large-eddy simulation, LES) stanowi alternatywę dla DNS.
W tej metodzie drobnoskalowa turbulencja nie jest bezpośrednio rozwiązywana, ale jest parametryzowana przy użyciu modelu podskalowego.
Umożliwia to modelowanie przepływów turbulentnych o większej liczbie Reynoldsa przy użyciu siatek o mniejszej liczbie węzłów.
Ograniczenie złożoności obliczeniowej odbywa się jednak kosztem mniejszej precyzji i fizycznej dokładności.
Niniejsza praca stanowi porównanie obu metod (DNS i LES), zarówno w kwestii fizycznej dokładności, jak i wydajności, skupiając się w szczególności na symulacjach z dwustronnym sprzężeniem pędu pomiędzy płynem i cząstkami. 
Zostało wykazane, że LES stanowi obiecującą odpowiedź na ograniczenia DNSu, pomimo pewnych niedokładności wynikających z filtrowania turbulencji w najmniejszych skalach.
Różnice te, w większości przypadków, stają się mniejsze, gdy uwzględni się dwustronne sprzężenie pędu, w szczególności dla cząstek poddanych działaniu grawitacji.
\newline \indent
Przeprowadzone zostały również analizy kosztów obliczeniowych obu metod.
Choć zastosowanie LES zapewnia ogromną oszczędność w obliczeniach związanych z turbulentnym przepływem płynu, to jednak koszt ten znacząco rośnie wraz z liczbą cząstek.
Niezbędna ilość symulowanych cząstek może zaś być duża, gdyż efekty dwustronnego sprzężenia pędu objawiają się wyraźniej w układach z względnie wysokim stosunkiem masy cząstek i płynu (pomiędzy $0.1$~i~$1$).
Zatem obliczenia odnoszące się do dynamiki cząstek stanowią bardziej istotne ograniczenie dla LESu niż dla DNSu, gdzie złożoność obliczeń związanych z~płynem jest czynnikiem dominującym.
Ponadto zidentyfikowano i przeanalizowano szereg parametrów, które pośrednio wpływają na czas trwania symulacji, takie jak np. promienie kropel oraz~działanie grawitacji.
Na~koniec przedstawiono także wpływ preferencyjnego koncentrowania się cząstek na wariancję ich rozkładu pomiędzy równolegle wykonywanymi procesami.
Ten czynnik ma~również istotne znaczenie w kontekście całkowitej wydajności symulacji.
}

\renewcommand{\abstractname}{Abstract}
\begin{abstract}
Turbulent fluid flows with dispersed particles are a common occurrence in nature, as~well~as in many technological processes.
An important example of such systems are atmospheric clouds that consist of a huge number of small water droplets and ice crystals suspended in air.
The~understanding of interplay between air (fluid) and aerosol (particles) is crucial for increasing the accuracy and reliability of numerical weather forecasts through more realistic modelling of atmospheric phenomena.
Performing numerical simulations on a fine scale is one way of studying such processes.

Direct numerical simulations (DNS) prove to be a relatively precise tool that is widely used in these studies.
DNS can be used to obtain the particle collision statistics obtained from these simulations may be further utilised as parameterisations for larger-scale models.
These simulations, however, are costly in terms of computational complexity, hence their application is limited to moderate-sized grids ($\sim 2048^3$ nodes, with the computing power of modern supercomputers).
For that reason DNS are not able to resolve turbulence characterised by~higher Reynolds numbers ($R_{\lambda} \sim 10^4$), which is observed in atmospheric air.  

Large-eddy simulations (LES) provide an alternative approach to DNS.
In this method small-scale turbulence is not directly resolved but parameterised using a subgrid-scale model.
This reduces the grid size required to achieve higher Reynolds numbers (and, thus, computational complexity) at the cost of diminished precision and physical fidelity. 
This thesis provides comparison of DNS and LES, both in terms of physical fidelity and performance, focusing primarily on simulations under two-way momentum coupling between the fluid and particles.
LES is shown to be a promising solution to limitations of DNS, even though it~exhibits certain inaccuracies that may be attributed to the filtering of small-scale vortical structures.
These discrepancies, for the most part, become less pronounced when two-way momentum coupling is considered, especially for settling particles.

Furthermore, a broader study of computational costs for both methods is presented.
Even~though LES provides immense performance advantage when it comes to fluid simulation, it~is~highly susceptible to the growing number of individually tracked particles.
The amount of particles that are needed may be considerable, as effects of two-way momentum coupling manifest when particle mass loadings are relatively high (between $0.1$ and $1$).
Thus, computations related to particles are recognised to be a much more significant bottleneck for~LES than for DNS, where the complexity of fluid simulation is the largest concern.
In~addition, several parameters that indirectly influence the execution time are identified and analysed, such as droplet radii or effects of gravity.
Finally, it is shown that the preferential concentration of particles may affect the variance of their distribution between parallel processes and~consequently influence the overall simulation performance.
 
\end{abstract}


\tableofcontents


\chapter*{Acknowledgements}
\addcontentsline{toc}{chapter}{\numberline{}Acknowledgements}
\label{ch:ack}

NNN



\chapter*{Introduction}
\addcontentsline{toc}{chapter}{\numberline{}Introduction}
\label{ch:int}

NNN


\chapter{Numerical Methods}
\label{ch:ch1}

NNN


\chapter{Physical Fidelity of DNS and LES Results}
\label{ch:ch2}

NNN



\chapter{Comparison of DNS and LES Performance}
\label{ch:ch3}

NNN



\chapter*{Conclusions}
\addcontentsline{toc}{chapter}{\numberline{}Conclusions}
\label{ch:end}

NNN


\appendix
\chapter{Pseudo-Spectral Method in Detail}
\label{app:psm}

NNN



\chapter{Effects of Subgrid-Scale Model on Flow Statistics}
\label{app:sgs}

NNN


\chapter{Super-Particle Parametrisations in Simulations under Two-Way Momentum Coupling}
\label{app:spp}

NNN


\addcontentsline{toc}{chapter}{\numberline{}References}
\printbibliography[title=References]

\end{document}
